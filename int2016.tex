\documentclass{beamer}

\usepackage{fontspec}
\setsansfont{Lato}

\definecolor{theme}{RGB}{28,90,127}
\definecolor{offblack}{HTML}{262626}
\setbeamercolor{normal text}{fg=offblack}

\usecolortheme[named=theme]{structure}
\usecolortheme{rose}  % inner
\usecolortheme{dolphin}  % outer

% modified version of default frametitle with horizontal separation line
\makeatletter
\setbeamertemplate{frametitle}{
  \ifbeamercolorempty[bg]{frametitle}{}{\nointerlineskip}%
  \@tempdima=\textwidth%
  \advance\@tempdima by\beamer@leftmargin%
  \advance\@tempdima by\beamer@rightmargin%
  \begin{beamercolorbox}[sep=0.3cm,left,wd=\the\@tempdima]{frametitle}
    \usebeamerfont{frametitle}%
    \vbox{}\vskip-2ex%
    \if@tempswa\else\csname beamer@fteleft\endcsname\fi%
    \strut\insertframetitle\strut\par%
    {%
      \ifx\insertframesubtitle\@empty%
      \else%
      {\usebeamerfont{framesubtitle}\usebeamercolor[fg]{framesubtitle}\insertframesubtitle\strut\par}%
      \fi
    }%
    \vskip.45ex%
    \hrule %height .6pt%
    \vskip-1.45ex%
    \if@tempswa\else\vskip-.3cm\fi%
  \end{beamercolorbox}%
}
\makeatother

% clean up footer
\beamertemplatenavigationsymbolsempty
\defbeamertemplate{footline}{custom footline}{
  \usebeamercolor[fg]{page number in head/foot}
  \usebeamerfont{page number in head/foot}
  \quad
  \insertshortauthor\enskip(\insertshortinstitute)
  \hfill
  \insertshorttitle
  \hfill
  \insertframenumber\,/\,\inserttotalframenumber\kern1em\vskip2pt
}
\setbeamertemplate{footline}[custom footline]

\useinnertheme{default}
\setbeamertemplate{itemize item}{\raise.35ex\hbox{\vrule width .7ex height .7ex}}
\setbeamertemplate{itemize subitem}{\raise.35ex\hbox{\vrule width .6ex height .6ex}}

% for backup slides
\usepackage{appendixnumberbeamer}

\usepackage{graphicx}
\graphicspath{{fig/}}

\usepackage{amsmath}
\usepackage{amssymb}
\usepackage{booktabs}
\usepackage{setspace}
\usepackage{tikz}
\usetikzlibrary{positioning}

\newcommand{\avg}[1]{\langle #1 \rangle}
\newcommand{\nch}{N_\text{ch}}
\newcommand{\vnk}[2]{v_#1\{#2\}}
\newcommand{\tran}{^\intercal}
\newcommand{\trento}{T\raisebox{-.5ex}{R}ENTo}
\newcommand{\order}[1]{$\mathcal O(10^{#1})$}
\newcommand{\x}{\mathbf x}
\newcommand{\y}{\mathbf y}
\newcommand{\z}{\mathbf z}
\newcommand{\xs}{\x_\star}
\newcommand{\zs}{\z_\star}
\newcommand{\yexp}{\y_\text{exp}}
\newcommand{\zexp}{\z_\text{exp}}

\newcommand{\arxivnum}{1605.03954}
\newcommand{\arxivclass}{nucl-th}
\newcommand{\conference}{INT workshop: Bayesian methods in nuclear physics}
\title
[Precision extraction of QGP properties (\arxivnum)]
{Precision extraction of QGP properties \\ with quantified uncertainties}
\subtitle{Part II: methodology and results}
\author[J.\ E.\ Bernhard, S.\ A.\ Bass]{Jonah E.\ Bernhard, Steffen A.\ Bass}
\institute[Duke]{Duke University}
\date{Wednesday, June 15, 2016}
\titlegraphic{\includegraphics[scale=2.3]{posterior_pair}}


\begin{document}


\section{Title}

\begin{frame}[plain,noframenumbering]
  \begin{tikzpicture}[remember picture, overlay]
    \node at (current page.center) {\inserttitlegraphic};
    \fill[white, opacity=.85] (current page.north west) rectangle (current page.south east);
    \node[anchor=south east] at (current page.south east) {\scriptsize arXiv:\arxivnum\ [\arxivclass]};
  \end{tikzpicture}
  \centering
  \setstretch{1.35}
  {\color{theme}{\Large\inserttitle}\\[1ex]\large\insertsubtitle} \\[2ex]
  \small
  \insertauthor \\[2ex]
  \scriptsize
  \conference \\
  \insertdate
\end{frame}


\section{Introduction}

\begin{frame}{Title}
  hello
\end{frame}


\section{Method}

\begin{frame}{Title}
  \begin{itemize}
    \item hello
    \item monkey
    \item aardvark
  \end{itemize}
\end{frame}

\begin{frame}{\trento\ initial condition model}
  \vspace{1em}
  \includegraphics{trento_events}
\end{frame}

\begin{frame}{Experiment design}
  \centering
  \includegraphics{design}
\end{frame}

\begin{frame}{Training data}
  \begin{tikzpicture}[remember picture, overlay]
    \node[yshift=-1em] at (current page.center) {
      \includegraphics{observables_prior}
    };
    \node[above left=1em of current page.center, text width=.5\textwidth] {
      \begin{itemize}
        \item Model calculations at each design point
        \item To be used as training data for emulator
      \end{itemize}
    };
  \end{tikzpicture}
\end{frame}

\begin{frame}{Gaussian process emulator}
  \includegraphics{gp}
\end{frame}

\begin{frame}{Multivariate output}
  \includegraphics{pca}
  \includegraphics{pca_variance}
\end{frame}

\begin{frame}{Validation}
  \includegraphics{validation}
\end{frame}

\begin{frame}{MCMC}

\end{frame}


\section{Results}

\begin{frame}[plain]
  \centering
  \includegraphics<1>{posterior_lower}
  \includegraphics<2>{posterior_pair}
\end{frame}

\begin{frame}{Estimate of $(\eta/s)(T)$}
  \centering
  \includegraphics{etas_estimate}
\end{frame}

\begin{frame}{Posterior samples}
  \begin{tikzpicture}[remember picture, overlay]
    \node[yshift=-1em] at (current page.center) {
      \includegraphics<1>{observables_prior}
      \includegraphics<2>{observables_posterior}
    };
    \node[
      below right=2em of current page.north west, inner ysep=3.5em,
      text width=.5\textwidth, align=center
    ] {
      Model calculations at each design point \\
      \only<2->{
        $\downarrow$ \\
        Emulator predictions from calibrated posterior
      }
    };
  \end{tikzpicture}
\end{frame}

\begin{frame}{Most probable parameters}
  \makebox[\textwidth]{
    \includegraphics<1>{mode_observables_id}
    \includegraphics<2>{mode_observables_both}
  }
\end{frame}


\section{Conclusion}

\begin{frame}{Summary}
  hello
\end{frame}

% \appendix


\end{document}
